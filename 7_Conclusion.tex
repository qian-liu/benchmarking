\section{Conclusion and the Future Work}
\label{sec:summ}
\subsection{What we have said and done}
%The Proposal: Advantages
%In this work we propose a set of image conversion methods, in particular we have performed a full conversion of the MNIST database. Our implementations are open-source and can be obtained from a public repository.
%
%In order to ease the understanding and comparison of investigation results, we suggest that researchers report typical neural networks characteristics as well as others that we believe are important (e.g. events per time unit, time per sample, response time).
%
%The use of neuromorphic hardware in research is increasing, some of its characteristics may alter simulation results. We invite researchers to describe some hardware characteristics that have a direct implication in the performance of neural networks.

This paper puts forward the NE dataset as a baseline for comparisons on vision based SNNs.
It aims to (1) promote meaningful comparison among algorithms in the field of neural computation, (2) allow comparison with conventional image recognition methods, (3) provide an assessment of the state of the art in spike-based visual recognition, and (4) help researchers identify future directions and advance the field.

The complementary evaluation methodology is essential to assess both the model-level and hardware-level performances.
For a SNN model, its network topology, neural and synapses models used and training methods are major descriptions while the recognition accuracy, network latency and also the time taken for both training and testing are important measurements of a spike-based model.
With an identical SNN model built on the dataset will benchmark various neuromorphic hardware to compare their performances.
Table 3.

Using the Poissonian subset of NE15-MNIST dataset, there are two benchmark systems proposed. 

\subsection{The future direction of developing the database}
What are the future algorithms we are encouraging.

