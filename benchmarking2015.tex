%%%%%%%%%%%%%%%%%%%%%%%%%%%%%%%%%%%%%%%%%%%%%%%%%%%%%%%%%%%%%%%%%%%%%%%%%%%%%%%%%%%%%%%%%%%%%%%%%%%%%%%%%%%%%%%%%%%%%%%%%%%%%%%%%%%%%%%%
% This is just a template to use when submitting manuscripts to Frontiers, it is not mandatory to use frontiers.cls nor frontiers.tex  %
%%%%%%%%%%%%%%%%%%%%%%%%%%%%%%%%%%%%%%%%%%%%%%%%%%%%%%%%%%%%%%%%%%%%%%%%%%%%%%%%%%%%%%%%%%%%%%%%%%%%%%%%%%%%%%%%%%%%%%%%%%%%%%%%%%%%%%%%


\documentclass[pdftex]{bioinfo}
\usepackage{subfig}
\usepackage{array}
\usepackage{amsmath}
\usepackage{amssymb}
\usepackage{url}
\usepackage{mathptmx}
\usepackage{acronym}
\usepackage{graphicx}
\usepackage{algorithm}
\usepackage{algpseudocode}
\usepackage[colorlinks]{hyperref} % in order to operate correctly, hyperref must be the last package declared
\usepackage{multirow}
\usepackage{makecell}
\usepackage{mathptmx}
\usepackage{amsmath}
%define some own functions
\newcommand{\tabincell}[2]{\begin{tabular}{@{}#1@{}}#2\end{tabular}} 
\def\D{\mathrm{d}}
% correct bad hyphenation here
\hyphenation{op-tical net-works semi-conduc-tor}
\pagestyle{empty}


\hypersetup{
  hypertexnames=true, 
  linkcolor=blue,anchorcolor=black,citecolor=blue,urlcolor=blue
}



\newenvironment{equationexp}[1]% Environment for explaining equation variables
{\begin{list}{}%
{\setlength{\leftmargin}{#1}}%
  \item[]%
}
{\end{list}}


\DeclareGraphicsExtensions{.jpg,.pdf,.mps,.png}
\graphicspath{{./images/}} % PUT ALL PDF/JPG/PNG FIGURES IN THIS SUBDIR

% correct bad hyphenation here
\hyphenation{op-tical net-works semi-conduc-tor}

%Commands useful for the review
\newcommand{\revised}[1]{{\color{red} #1}}
\newcommand{\SC}[1]{{\color{red} \textbf{SC: } #1}}


\setlength{\abovecaptionskip}{5pt}
\setlength{\belowcaptionskip}{5pt}

\copyrightyear{2015}
\pubyear{2015}


%%%%%
%\documentclass{frontiersENG} % for Engineering articles
%%\documentclass{frontiersSCNS} % for Science articles
%%\documentclass{frontiersMED} % for Medicine articles
%
%\usepackage{url,lineno}
%\usepackage{epstopdf}
%%\IEEEoverridecommandlockouts
%\usepackage{graphicx}
%\usepackage{subfigure}
%\usepackage{xr-hyper}
%\usepackage{hyperref}
%\linenumbers
%
%% Leave a blank line between paragraphs in stead of using \\
%
%\copyrightyear{}
%\pubyear{}

%%%%%%
%\def\journal{NEUROMORPHIC ENGINEERING}%%% write here for which journal %%%
%\def\DOI{}
%\def\articleType{Research Article}
%\def\keyFont{\fontsize{8}{11}\helveticabold }
\def\firstAuthorLast{Qian Liu {et~al.}} %use et al only if is more than 1 author
%\def\Authors{Qian Liu, Garibaldi Pineda Garcia, Evangelos Stromatias, Teresa Gotarredona, and Steve Furber}
\def\Authors{Qian~Liu\,$^{1,*}$, Garibaldi~Pineda~García\,$^{1}$, Evangelos~Stromatias\,$^{1}$, Teresa~Gotarredona\,$^{2}$, and Steve~Furber\,$^{1}$}
% Affiliations should be keyed to the author's name with superscript numbers and be listed as follows: Laboratory, Institute, Department, Organization, City, State abbreviation (USA, Canada, Australia), and Country (without detailed address information such as city zip codes or street names).
% If one of the authors has a change of address, list the new address below the correspondence details using a superscript symbol and use the same symbol to indicate the author in the author list.
%\def\Address{SpiNNaker, Advanced Processor Technologies Research Group, School of Computer Science, University of Manchester, Manchester, United Kingdom}
\def\Address{$^{1}$SpiNNaker, Advanced Processor Technologies Research Group, School of Computer Science, University of Manchester, Manchester, United Kingdom\\
$^{2}$Instituto de Microelectrónica de Sevilla (IMSE-
CNM-CSIC), Sevilla, Spain }
%% The Corresponding Author should be marked with an asterisk
%% Provide the exact contact address (this time including street name and city zip code) and email of the corresponding author
\def\corrAuthor{Qian Liu}
\def\corrAddress{SpiNNaker, Advanced Processor Technologies Research Group, School of Computer Science, The University of Manchester, Oxford Road, Manchester, M13 9PL, United Kingdom}
\def\corrEmail{qianl.liu-3@manchester.ac.uk}
%
%% \color{FrontiersColor} Is the color used in the Journal name, in the title, and the names of the sections
%

\begin{document}
\firstpage{1}

\title[Benchmarking Spike-Based Visual Recognition: a Dataset and Evaluation]{Benchmarking  Spike-Based Visual Recognition:\\ a Dataset and Evaluation}
\author[\firstAuthorLast ]{\Authors}
\address{\Address}
%\correspondance{}
\history{}

\editor{}


\maketitle
\begin{abstract}
To gain a better understanding of the brain and build biologically-inspired computers, increasing attention is being paid to research into spike-based neural computation.
Within the field, the visual pathway and its hierarchical organisation have been extensively studied within the primate brain.
Spiking Neural Networks (SNNs) inspired by the understanding of observed biological structure and functions have been successfully applied to visual recognition/classification tasks.
A new series of vision benchmarks for spike-based neural processing are now needed to quantitatively measure progress within this rapidly advancing field.
We propose that a large dataset of spike-based visual stimuli is needed to provide a baseline for comparisons.
Furthermore a complementary evaluation methodology is also crucial to assess the accuracy and efficiency of an algorithm.

First of all, an initial NE (Neuromorphic Engineering) dataset of input stimuli based on standard computer vision benchmarks consisting of %facial images (FERET database) and 
digits (MNIST database) is presented according to the current research on spike-based image recognition.
Within this dataset, all images are centre aligned and with similar scale.
We describe how we intend to expand this dataset to fulfil the needs of upcoming research problems.
For instance, the data should provide cases to measure position-, scale-, and viewing-angle invariance.
The data will be in Address-Event Representation (AER) format which is well-applied in neuromorphic engineering field unlike conventional images.
These spike trains are produced by various techniques: rate-based Poisson spike generation, rank order encoding and recorded output from a silicon retina with both flashing and oscillating input stimuli.
An evaluation methodology is also presented which describes how to consistently assess the accuracy, speed, efficiency and cost of an algorithm working with the dataset.
Finally, we provide a baseline for comparison based on a proposed SNN's performance on the dataset.
The network is trained on-line using the Spike Timing Dependent Plasticity (STDP) learning rule on a massive-parallel neuromorphic simulator, e.g. SpiNNaker.

With this benchmark we hope to (1) promote meaningful comparison among algorithms in the field of neural computation, (2) allow comparison with conventional image recognition methods, (3) provide an assessment of the state of the art in spike-based visual recognition, and (4) help researchers identify future directions and advance the field.

\tiny
\section{Keywords:} Benchmarking, Neuromorphic Engineering, Real-Time, Spiking Neural Networks, Vision
%All article types: you may provide up to 8 keywords; at least 5 are mandatory.
\end{abstract}

Outline:
\begin{itemize}
	\item present a dataset
	\item evaluate models on the dataset(SW/HW)
	\item examples of using the dataset and evaluation
\end{itemize}
%\label{sec:intro}
\section{Introduction}
\label{sec:intro}
%\subsection{What Is the Problem}
With the rapid development on neural engineering, researchers are approaching the aims of understanding the brain functions and building brain-like machines upon the knowledge~\cite{furber2007neural}. 
As a fast growing field, neuromorphic engineering has provided biologically-inspired sensors such as DVS(Dynamic Vision Sensor) silicon retinas~\cite{serrano-gotarredona_128_2013, lichtsteiner2008128}, which are good examples of low-cost visual processing thanks to their event-driven and redundancy-reducing style of computation.
Moreover, SNNs simulation tools~\cite{davison2008pynn, gewaltig2007nest, goodman2008brian} and neuromorphic hardware platforms~\cite{furber2013overview,  schemmel2010wafer, moradi2014event} have been developed to allow exploration of the brain by mimicking its functions and developing large-scale practical applications.
SNN has become an active area of computer vision thanks to the explicit  biological study on the central visual pathway.
The central visual system consists of several cortical areas responsible for visual processing, which are placed in a hierarchical pattern according to the anatomical experiments~\cite{felleman1991distributed}.
Fast object recognition proceeds in  the feed-forward hierarchy of the ventral pathway, and the information is unfolded along the stream to the  IT (Inferior Temporal) cortex~\cite{dicarlo2012does}.
%There are two basic streams locating in the visual area: a dorsal and a ventral pathway.
%They differ in behavioural patterns according to the observation from brain lesions~\cite{prado2005two}, and also in functions where the ventral (`perception') stream perceives the world by means of object recognition and memory, while the dorsal (`action') stream provides real-time visual guidance for motor actions such as eye movements and grasping objects~\cite{goodale1992separate}. 

Riesenhuber and Poggio (\cite{riesenhuber1999hierarchical}) proposed a quantitative modelling framework of object recognition with position-, scale- and view-invariance based on the units of MAX-like operations.
The cortical-like model has been analysed on several datasets~\cite{serre2007robust}.
And recently Fu and etc. (\cite{fu_spiking_2012}) reported their SNN implementation of the framework was capable of facial expression recognition tasks (97.35\% on JAFFE dataset~\cite{lyons1998coding} which contains 213 images of 7 facial expressions posed by 10 individuals).
% 97.35\% on JAFFE dataset.
They employed simple integrate-and-fire neurons with rank order coding (ROC) where  the earliest pre-synaptic spikes have the strongest impact on the post synaptic potentials.
According to Thorpe~\cite{vanrullen_surfing_2002}, the first wave of spikes  carry explicit information through the ventral stream and in each stage meaningful information is extracted and spikes are regenerated. 
Using one spike per neuron, Delorme and Thorpe~\cite{delorme_face_2001} reported 100\% and 97.5\% of accuracies on the face identification task over changing  contrast and luminance training (40 individuals $\times$ 8 images) and testing data  (40 individuals $\times$ 2 images) respectively.
%These developments yielded a large number of papers on SNNs based recognition, with a majority reporting outstanding recognition resulton limited-size databases.

Convolutional Neural Network (CNN), also known as \textit{ConvNet} which is carried out by Lecun and Bengio (\cite{lecun1995convolutional}), is a well applied model of such a cortical-like framework.
%Reported results:
%Hand Gestures, Qian Liu
An early Convolutional Spiking Neural Network (CSNN) model identified faces of 35 persons with an accuracy of 98.3\% exploiting simple integrate and fire neurons~\cite{matsugu2002convolutional}.
Another CSNN model~\cite{zhao_feedforward_2014} was trained and tested both with DVS raw data and Leaky Integrate-and-Fire (LIF) neurons.
%The MAX operation, training and the switch are not only neuron involved.
It was capable of recognising three moving postures with an accuracy about 99.48\% and 88.14\% on the MNIST-DVS dataset (see Chapter~\ref{sec:data}).
As one step forward, the paper (\cite{camunas2012event}) implemented a convolution processor module in hardware which could be combined with an DVS for high-speed recognition tasks.
The input of the ConvNet was continuous spike events instead of static images or frame-based videos. 
The chip detected four suits of a 52 card deck while the cards were fast browsed in only 410 ms.
Similarly, a real-time gesture recognition model~\cite{liu2014real} was implemented on a neuromorphic system with a DVS as a front-end and a SpiNNaker~\cite{furber2013overview} machine as the back-end where LIF neurons built up the ConvNet configured with biological parameters.
In this study's largest configuration, a network of 74,210 neurons and 15,216,512 synapses used 290 SpiNNaker cores in parallel and reached 93.0\% accuracy. 

Deep neural networks together with deep learning are the most exciting research fields in vision recognition.
Spiking deep network is of great potential to combine the remarkable performance and a more energy efficient training and running.
In the initial stage of the research, the study was focused on converting off-line trained deep network to SNNs~\cite{o2013real}.
The performance was increased from 92.0\% to 95.0\%~\cite{unpublished_EVAN} by implementing the model on SpiNNaker instead of the earlier FPGA version.
Recent attempts have been contributed to better translation by utilising modified units in ConvNet (\cite{cao2015spiking}) and tuning the weights and thresholds (\cite{unpublished_peter}).
They both accomplished state-of-the-art performance (\% and 99.1\% on MNIST dataset) comparing to original SNNs.
Current trend of training Spiking DNNs on-line using biologically-plausible learning methods is also promising.
An event driven Contrastive Divergence (CD) training algorithm for SNNs was proposed and verified on MNIST (91.9\%).


%Hand Gestures, Lee
The motion trajectories of moving hands are detected by spatiotemporally correlating the stereoscopically verged asynchronous events from the DVSs by using leaky integrate-and-fire (LIF) neurons. Adaptive thresholds of the LIF neurons achieve the segmentation of trajectories, which are then translated into discrete and finite feature vectors. The feature vectors are classified with hidden Markov models, using a separate Gaussian mixture model for spotting irrelevant transition gestures. The disparity information from stereovision is used to adapt LIF neuron parameters to achieve recognition invariant of the distance of the user to the sensor, and also helps to filter out movements in the background of the user. Exploiting the high dynamic range of DVSs, furthermore, allows gesture recognition over a 60-dB range of scene illuminance. The system achieves recognition rates well over 90\% under a variety of variable conditions with static and dynamic backgrounds with naïve users.
%MNIST unsupervised STDP unpublished
%Presented in Poissonian spike trains, the MNIST hand-written digits could be classified correctly with 95.0\% accuracy based on STDP unsupervised learning.
%The performance was achieved by 6,400 classifier neurons and presenting 15 times the entire MNIST training set.
%MNSIT Emre Neftci
%The recurrent activity of the network replaces the discrete steps of the CD algorithm, while Spike Time Dependent Plasticity (STDP) carries out the weight updates in an online, asynchronous fashion. We demonstrate our approach by training an RBM composed of leaky I&F neurons with STDP synapses to learn a generative model of the MNIST hand-written digit dataset, and by testing it in recognition, generation and cue integration tasks. Our results contribute to a machine learning-driven approach for synthesizing networks of spiking neurons capable of carrying out practical, high-level functionality.
%MNIST Evangelos Stromatias
In  this  context  we  in-
troduce  a  realization  of  a  spike-based  variation  of  previously
trained  DBNs  on  the  biologically-inspired  parallel  SpiNNaker
platform. The DBN on SpiNNaker runs in real-time and achieves
a classification performance of 95% on the MNIST handwritten
digit  dataset,  which  is  only  0.06%  less  than  that  of  a  pure
software implementation. Importantly, using a neurally-inspired
architecture yields additional benefits: during network run-time
on this task, the platform consumes only 0.3 W with classification
latencies in the order of tens of milliseconds, making it suitable
for  implementing  such  networks  on  a  mobile  platform.  The
results in this paper also show how the power dissipation of the
SpiNNaker platform and the classification latency of a network
scales with the number of neurons and layers in the network and
the overall spike activity rate.


Few of these algorithms reported results on images from a common database; fewer met the desirable goal of being evaluated against a standard testing protocol that includes separate training and testing sets.
As a consequence, there was no way to make a quantitative assessment of the algorithms' relative strengths and weaknesses.
Unfortunately, this is not an isolated case, but an endemic problem in computer vision research.



\subsection{The Proposal: Advantages}
In this paper we present a comprehensive method of eval-
uating face-recognition
algorithms, developed as part of the
Face Recognition Technology (FERET) program [ 10,111.
The FERET evaluation methodology consists of an inte-
grated data collection effort and testing program. These
two parts are integrated through the FERET database of
facial images; the database is divided into a development
portion, which is provided to researchers, and a sequestered
portion for testing. The partition of the database enables
algorithms to be trained and tested on different, but related,
sets of images. The FERET evaluation procedure is a set of
standard testing protocols: the FERET tests are indepen-
dently administered
and each test is completed within
three days. The use of a standard testing protocol allows
for the direct comparison of algorithms developed by dif-
ferent groups, as well as for measuring improvements made
by any single group over time.

First of all, an initial dataset of input stimuli based on standard computer vision benchmarks consisting of %facial images (FERET database) and 
digits (MNIST database) is presented according to the current research on spike-based image recognition.
Within this dataset, all images are centre aligned and with similar scale.
We describe how we intend to expand this dataset to fulfil the needs of upcoming research problems.
For instance, the data should provide cases to measure position-, scale-, and viewing-angle invariance.
The data will be in Address-Event Representation (AER) format which is well-applied in neuromorphic engineering field unlike conventional images.
These spike trains are produced by various techniques: rate-based Poisson spike generation, rank order encoding and recorded output from a silicon retina with both flashing and oscillating input stimuli.
An evaluation methodology is also presented which describes how to consistently assess the accuracy, speed, efficiency and cost of an algorithm working with the dataset.
Finally, we provide a baseline for comparison based on a proposed SNN's performance on the dataset.
The network is trained on-line using the Spike Timing Dependent Plasticity (STDP) learning rule on a massive-parallel neuromorphic simulator, e.g. SpiNNaker.
\section{Related Work}
\label{sec:Related}
\subsection{Vision Literature: Related Work}
\subsubsection{MNIST~[\cite{lecun_gradient-based_1998}]}
The MNIST database was constructed from NIST's Special Database 3 and Special Database 1 which contain binary images of handwritten digits. NIST originally designated SD-3 as their training set and SD-1 as their test set. However, SD-3 is much cleaner and easier to recognize than SD-1. The reason for this can be found on the fact that SD-3 was collected among Census Bureau employees, while SD-1 was collected among high-school students. Drawing sensible conclusions from learning experiments requires that the result be independent of the choice of training set and test among the complete set of samples. Therefore it was necessary to build a new database by mixing NIST's datasets.

The MNIST training set is composed of 30,000 patterns from SD-3 and 30,000 patterns from SD-1. Our test set was composed of 5,000 patterns from SD-3 and 5,000 patterns from SD-1. The 60,000 pattern training set contained examples from approximately 250 writers. We made sure that the sets of writers of the training set and test set were disjoint.

SD-1 contains 58,527 digit images written by 500 different writers. In contrast to SD-3, where blocks of data from each writer appeared in sequence, the data in SD-1 is scrambled. Writer identities for SD-1 is available and we used this information to unscramble the writers. We then split SD-1 in two: characters written by the first 250 writers went into our new training set. The remaining 250 writers were placed in our test set. Thus we had two sets with nearly 30,000 examples each. The new training set was completed with enough examples from SD-3, starting at pattern 0, to make a full set of 60,000 training patterns. Similarly, the new test set was completed with SD-3 examples starting at pattern 35,000 to make a full set with 60,000 test patterns. Only a subset of 10,000 test images (5,000 from SD-1 and 5,000 from SD-3) is available on this site. The full 60,000 sample training set is available.

Many methods have been tested with this training set and test set. Here are a few examples. Details about the methods are given in an upcoming paper. Some of those experiments used a version of the database where the input images where deskewed (by computing the principal axis of the shape that is closest to the vertical, and shifting the lines so as to make it vertical). In some other experiments, the training set was augmented with artificially distorted versions of the original training samples. The distortions are random combinations of shifts, scaling, skewing, and compression. 

\subsubsection{ImageNet~[\cite{deng_imagenet:_2009}]}
 What is ImageNet?
ImageNet is an image dataset organized according to the WordNet hierarchy. Each meaningful concept in WordNet, possibly described by multiple words or word phrases, is called a "synonym set" or "synset". There are more than 100,000 synsets in WordNet, majority of them are nouns (80,000+). In ImageNet, we aim to provide on average 1000 images to illustrate each synset. Images of each concept are quality-controlled and human-annotated. In its completion, we hope ImageNet will offer tens of millions of cleanly sorted images for most of the concepts in the WordNet hierarchy.
Why ImageNet?
The ImageNet project is inspired by a growing sentiment in the image and vision research field – the need for more data. Ever since the birth of the digital era and the availability of web-scale data exchanges, researchers in these fields have been working hard to design more and more sophisticated algorithms to index, retrieve, organize and annotate multimedia data. But good research needs good resource. To tackle these problem in large-scale (think of your growing personal collection of digital images, or videos, or a commercial web search engine’s database), it would be tremendously helpful to researchers if there exists a large-scale image database. This is the motivation for us to put together ImageNet. We hope it will become a useful resource to our research community, as well as anyone whose research and education would benefit from using a large image database. 
%Current http://www.image-net.org/
Currently there are 14,197,122 images, 21841 synsets indexed in the dataset. 
\subsubsection{Microsoft COCO~[\cite{lin_microsoft_2014}]}
We introduce a new large-scale dataset that addresses
three core research problems in scene understanding: de-
tecting non-iconic views (or non-canonical perspectives
[12]) of objects, contextual reasoning between objects
and the precise 2D localization of objects. For many
categories of objects, there exists an iconic view. For
example, when performing a web-based image search
for the object category “bike,” the top-ranked retrieved
examples appear in profile, unobstructed near the cen-
ter of a neatly composed photo. We posit that current
recognition systems perform fairly well on iconic views,
but struggle to recognize objects otherwise – in th
background, partially occluded, amid clutter [13] – re-
flecting the composition of actual everyday scenes. We
verify this experimentally; when evaluated on everyday
scenes, models trained on our data perform better than
those trained with prior datasets. A challenge is finding
natural images that contain multiple objects. The identity
of many objects can only be resolved using context, due
to small size or ambiguous appearance in the image. To
push research in contextual reasoning, images depicting
scenes [3] rather than objects in isolation are necessary.
Finally, we argue that detailed spatial understanding of
object layout will be a core component of scene analysis.
An object’s spatial location can be defined coarsely using
a bounding box [2] or with a precise pixel-level segmen-
tation [14], [15], [16]. As we demonstrate, to measure
either kind of localization performance it is essential
for the dataset to have every instance of every object2
category labeled and fully segmented. Our dataset is
unique in its annotation of instance-level segmentation
masks.
\subsubsection{YouTube Action Dataset~[\cite{liu_recognizing_2009}]}
In this paper, we present a systematic framework for recognizing realistic actions from videos “in the wild”. Such unconstrained videos are abundant in personal collections as well as on the Web. Recognizing action from such videos has not been addressed extensively, primarily due to the tremendous variations that result from camera motion, background clutter, changes in object appearance, and scale, etc. The main challenge is how to extract reliable and informative features from the unconstrained videos. We extract both motion and static features from the videos. Since the raw features of both types are dense yet noisy, we propose strategies to prune these features. We use motion statistics to acquire stable motion features and clean static features. Furthermore, PageRank is used to mine the most informative static features. In order to further construct compact yet discriminative visual vocabularies, a divisive information-theoretic algorithm is employed to group semantically related features. Finally, AdaBoost is chosen to integrate all the heterogeneous yet complementary features for recognition. We have tested the framework on the KTH dataset and our own dataset consisting of 11 categories of actions collected from YouTube and personal videos, and have obtained impressive results for action recognition and action localization.

Two early benchmarks are the KTH [32] and Weiz-
mann [1] sets, both used extensively over the years. These
sets provide low resolution videos of a few, “atomic” action
categories, such as walking, jogging, and running. These
videos were produced “in the lab”, and present actors that
perform scripted behavior. The videos they provide were
acquired under controlled conditions, with static cameras
and static, un-clutered backgrounds. In addition, actors ap-
pear without occlusions, thus allowing action recognition to
be performed by considering silhouettes alone [1].

In an effort to increase the diversity of appearances and
viewing conditions, benchmark designers have turned to
TV, sports broadcasts and motion pictures as sources for
challenging videos of human actions. These benchmarks
no longer represent controlled conditions; viewpoints, illu-
minations, occlusions are all arbitrary, thereby significantly
raising the bar for action recognition systems.
One early example is the UIUC2 benchmark [36], which
provided unconstrained sports videos of badminton matches
downloaded from YouTube. To our knowledge, this is the
first benchmark to provide such unconstrained data. An-
other early, popular example is the UCF-Sports bench-
mark [28] with its 200 videos of nine different sports events,
collected from TV broadcasts. Sports videos were also con-
sidered in the Olympic-Games benchmark of [23].

\subsection{Current Status}
what are the problems the current database aim for.
Current methods/algorithms existing.


The FERET tests are administered using a testing proto-
col, which states the mechanics of the tests and the manner
in which the test will be scored. In face recognition, for
example, the protocol states the number of images of each
person in the test, how the output from the algorithm is
recorded, and how the performance results are reported.
There is a direct connection among the problem being
evaluated, the images in the database, and the testing pro-
tocol. The testing protocol and the images define the pro-
blem to be evaluated. The characteristics and quality of the
images are major factors in determining the difficulty of the
problem. For example, if the faces are in a predetermined
position in the images, the problem is different from that for
images in which the faces can be located anywhere in the
image. In the FERET database, variability was introduced
by the inclusion of images taken at different dates and loca-
tions (see Section 4.2). This resulted in changes in lighting,
scale, and background.
The goals for the FERET evaluation process were to assess
the state of the art, advance the state of the art, and point to
future directions of research. Accomplishing all three goals
was a delicate process, and the keys to success were the
database and the tests. If algorithms existed that could easily
solve the problem, then the evaluation process would be
reduced to ‘tuning’ existing algorithms. On the other hand,
if the images defined a problem that was beyond current
algorithmic techniques, then the results would have been
poor and would have not allowed an accurate assessment of
current algorithmic capabilities. The key was to find the right
balance, so that if the problem formulated could not be solved
satisfactorily by existing methods, it would be possible to
develop algorithms that could solve it.
The collection of the FERET database was initiated in
September 1993, and the first FERET test was administered
in August 1994. A standard database of facial images was
established in the first year and made available to research-
ers; this database provided the images for the Aug94
FERET test. (Throughout this article, date-based names
such as Aug94 are used to refer to the same FERET test,
even when the tests were administered on other dates.) The
Aug94 FERET test established a performance baseline for
fully automatic face-recognition
algorithms. A fully auto-
matic algorithm does not require the location of the face in
the image as input: the algorithm locates and identifies the
face in the image.
The Aug94 FERET test was designed to measure the
performance of algorithms that could automatically locate,
normalize, and identify faces from a database. The test
consisted of three subsets: the large gallery, false alarm,
and rotation tests. The first tested the ability of algorithms
to recognize faces from a set of 317 known individuals
(referred to as the gallery; an image of an unknown face
presented to the algorithm is a probe, and the collection of
probes is called the probe set). The second subtest, the false-
alarm test, measured how well an algorithm rejects faces not
in the gallery. The third baselined the effects of pose
changes (rotations) on performance. On the basis of the
Aug94 FERET test, it was concluded that algorithms needed
to be evaluated on (1) larger galleries and (2) a substantially
increased number of duplicate probes. (A duplicate is
defined as an image of a person whose corresponding
gallery image was taken on a different date.)
A second FERET test, first administered in March 1995
(and referred to as the Mar95 FERET test), was designed
based on the conclusions from the Aug94 FERET test. The
Mar95 FERET test evaluated algorithms on larger galleries
and probe sets with a greater number of duplicates. This
required that additional images be collected, with an emphasis
on images of the same people taken months or years apart.

\subsubsection{Converting from MPL to SNN}
It remains a challenge to transform traditional artificial neural networks into spiking ones.
There are attempts~\cite{la2008response}~\cite{burkitt2006review} to estimate the output firing rate of the LIF neurons (Equation~\ref{equ:lif}) under certain conditions. 
%For the model illustrated above, there are two types of synaptic connection: one-to-one connections in the retina layer and N-to-one connections in all the convolutional layers (the pooling layer is also included). 
%For the retina layer, 1) the problem is: what is the connection weight between two single LIF neurons to make a post-synaptic neuron fire whenever the pre-synaptic neuron generates a spike? 
%While for the convolutional neurons, 2) given the input spike rates, LIF neuron parameters and the output spiking rate, what are the corresponding weights between the two layers?
\begin{equation}
\frac{\D \: V(t)}{\D\:  t}=-\frac{V(t)-V_\mathit{rest}}{\tau_m}+\frac{I(t)}{C_m}
\label{equ:lif}
\end{equation}
The membrane potential $V$ changes in response to input current $I$, starting at the resting membrane potential  $V_{rest}$, where the membrane time constant is $\tau_m = R_mC_m$, $R_m$ is the membrane resistance and $C_m$ is the membrane capacitance.

Given a constant current injection $I$, the response function, i.e. firing rate, of the LIF neuron is
\begin{equation}
\lambda_\mathit{out}=
\left [ t_\mathit{ref}-\tau_m\ln \left ( 1-\frac{V_{th}-V_\mathit{rest}}{IR_m}  \right )\right ]^{-1}
\label{equ:consI}
\end{equation}
when $IR_m>V_{th}-V_{rest}$, otherwise the membrane potential cannot reach the threshold $V_{th}$ and the output firing rate is zero. 
The absolute refractory period $t_\mathit{ref}$ is included, where all input during this period is invalid.
In a more realistic scenario, the post-synaptic potentials (PSPs) are triggered by the spikes generated from the neuron's pre-synaptic neurons other than a constant current.
Assume that the synaptic inputs are Poisson spike trains, the membrane potential of the LIF neuron is considered as a diffusion process. Equation~\ref{equ:lif} can be modelled as a stochastic differential equation referring to Ornstein-Uhlenbeck process,
\begin{equation}
\tau_m\frac{\D\:V(t)}{\D\:  t}=-\left[V(t)-V_\mathit{rest}\right] + \mu + \sigma\sqrt{2\tau_m}\xi (t)
\label{equ:sde}
\end{equation}
where
\begin{equation}
\begin{array}{l}
\mu=\tau_m(\mathbf{w_E\cdot\lambda_E}-\mathbf{w_I\cdot\lambda_I})
\\
\\
\sigma ^{2} = \frac{\tau_m}{2}\left(\mathbf{w_E^{2}\cdot\lambda_E}+\mathbf{w_I^{2}\cdot\lambda_I}\right)
\end{array}
\label{equ:ou}
\end{equation}
are the conditional mean and variance of the membrane potential.
The delta-correlated process $\xi(t)$ is Gaussian white noise with zero mean, $\mathbf{w_E}$ and $\mathbf{w_I}$ stand for the weight vectors of the excitatory and the inhibitory synapses, and $\mathbf{\lambda}$ represents the vector of the input firing rate.
The response function of the LIF neuron with Poisson input spike trains is given by the Siegert function~\cite{siegert1951first}, 
%\begin{equation}
%%\lambda_\mathit{out}=\left(\tau_\mathit{ref} + \frac{\tau_Q}{\sigma_Q}\sqrt{\frac{\pi}{2}} \int_{V_\mathit{rest}}^{V_\mathit{th}}du \:\exp \left(\frac{u-\mu_Q}{\sqrt2\sigma_Q} \right )^{2}\left[1+erf \left(\frac{u-\mu_Q}{\sqrt2\sigma_Q} \right ) \right ]\right)^{-1}
%\begin{split}
%\lambda_\mathit{out}=\left(\tau_\mathit{ref} + \frac{\tau_Q}{\sigma_Q}\sqrt{\frac{\pi}{2}} \int_{V_\mathit{rest}}^{V_\mathit{th}}\D u \:\exp \left(\frac{u-\mu_Q}{\sqrt2\sigma_Q} \right )^{2}\left[1+\mathrm{erf} \left(\frac{u-\mu_Q}{\sqrt2\sigma_Q} \right ) \right ]\right)^{-1}
%\end{split}
%\label{equ:sgt}
%\end{equation}
\begin{align}
\lambda_\mathit{out} &=\left(\tau_\mathit{ref} + \frac{\tau_Q}{\sigma_Q}\sqrt{\frac{\pi}{2}} \int_{V_\mathit{rest}}^{V_\mathit{th}}\D\,u \:\exp \left(\frac{u-\mu_Q}{\sqrt2\sigma_Q} \right )^{2} \right. \nonumber \\
&\qquad \left. \vphantom{\int_t} \cdot  \left[1+\mathrm{erf} \left(\frac{u-\mu_Q}{\sqrt2\sigma_Q} \right ) \right ]\right)^{-1}
\label{equ:sgt}
\end{align}
where $\tau_Q, \mu_Q, \sigma_Q$ are identical to $\tau_m, \mu, \sigma$ in Equation~\ref{equ:ou}, and erf is the error function.

Still there are some limitations on the response function. 
For the diffusion process, only small amplitude (weight) of the PostSynaptic Potentials (PSPs) generated by a large amount of input spikes (high spiking rate) work under this circumstance; 
plus, the delta function is required, i.e. the synaptic time constant is considered to be zero. Thus only a rough approximation of the output spike rate has been determined.
Secondly, given different input spike rate to each pre-synaptic neurons, the parameters of the LIF neuron and the output spiking rate, how to tune every single corresponding synaptic weight remains a difficult task.

\subsubsection{Rank-Order-Coding}

\subsubsection{Liquid State Machine/Reservoir Encoding}

\section{Guiding Principles}
\label{sec:guide}
\subsection{The Goals}
With this benchmark we hope to (1) promote meaningful comparison among algorithms in the field of neural computation, (2) allow comparison with conventional image recognition methods, (3) provide an assessment of the state of the art in spike-based visual recognition, and (4) help researchers identify future directions and advance the field.
\subsection{Dataset and Testing Protocols Referring the Goals}

%The FERET database was established to support both
%algorithm development and evaluation. Two guiding prin-
%ciples were followed. First, the evaluation of algorithms
%requires a common database of images for both develop-
%ment and testing. In the FERET evaluation, the images in
%the test are from both the development
%and sequestered
%portions of the FERET database. Second, the variety and
%difficulty of the problems defined by the images in the data-
%base must increase incrementally.
%The need to test algorithms against a database is obvious,
%but the development
%function of the database is equally
%important (if less obvious). For the evaluation procedure
%to produce meaningful results, the images in the develop-
%mental portion of the database must resemble those on
%which algorithms are to be tested. The development and
%testing data sets must be similar in both quality and quantity.
%For example, if the test will consist of a gallery of 1000
%individuals,
%it is not appropriate
%for the development
%database to consist of 50 individuals. The algorithms tested
%will be only as good as the database from which they are
%developed. The FERET evaluation procedure followed this
%principle by partitioning the FERET database into the devel-
%opmental and sequestered portions, where the developmen-
%tal portion was representative
%of the sequestered portion
%(details are provided in Section 4.2).
%The other principle is that the problem defined by the
%images in the database must mesh with the current level
%of algorithm development, and the difficulty of the database
%must grow as the sophistication of the algorithms increases.
%As explained in Section 2, if the database defines a problem
%that is too easy, testing the algorithm becomes a mere tuning
%exercise. At the other extreme, if the problem is too far
%beyond the state of the art, the test will not produce any
%meaningful results. To prevent the FERET database from
%becoming stale, we continuously expanded and adjusted the
%database to the state of the art in face recognition.



\section{The Database: NeuroMorphic15}
\label{sec:data}

\subsection{Description}
	Experiment setup/ collection method/ properties of each class/ etc.
	\subsubsection{Poisson}
	\subsubsection{Rank-Order-Encoding}
  A different way of encoding spikes is using a \emph{rank-order}; this means
keep just the order in which those spikes were fired and disregard the exact timing. Rank-ordered spike trains have been used in vision tasks under a biological plausibility constraint, making them a viable way of image encoding for neural applications~[\cite{van-rullen-rate-coding,basab-model,Masmoudi2010}].

Rank-ordered encoding is performed using an algorithm known as 
\emph{Filter overlap Correction algorithm} or \textbf{FoCal}~[\cite{basab-model}]. It models the \emph{foveal pit} region, the highest resolution area of the retina, with four ganglion cell layers that show a centre-surround behaviour~[\cite{Kolb2003}]. In order to simulate these layers, four discrete 2D \emph{convolutions} are performed. The centre-surround behaviour of the ganglion cells is modelled using Differences of Gaussians (DoG, Eq.~\ref{eq-dog}). 
\begin{equation}
\label{eq-dog}
DoG_w(x,y) = \pm\frac{1}{2\pi\sigma_{w,c}^2}e^{\frac{-(x^2 + y^2)}{2\sigma_{w,c}^2}}
\mp\frac{1}{2\pi\sigma_{w,s}^2}e^{\frac{-(x^2 + y^2)}{2\sigma_{w,s}^2}}
\end{equation}
where $\sigma_{w,c}$ and $\sigma_{w,s}$ are the standard deviation for the 
centre and surround components of the DoG at scale $w$ (cell type). The signs 
will be ($-$,$+$) if the ganglion cell has an \emph{off-centre} behaviour and 
($+$,$-$) if it has an \emph{on-centre} one. Table~\ref{tab-kernel-specs} 
describes the parameters used to compute the convolution \emph{kernels} at each 
scale $w$.

\begin{table}[htb]
 \caption{Simulation parameters for ganglion cells}
  \begin{center}


  \bgroup
  \def\arraystretch{1.4}
      
  \begin{tabular}{c c c c c c}
    \begin{minipage}{0.7cm}\centering Layer \end{minipage}& 
    \begin{minipage}{0.8cm}\centering Centre \\type \end{minipage}& 
    \begin{minipage}{0.7cm} \centering Matrix width \end{minipage}&  
    \begin{minipage}{1.3cm}\centering Centre std. dev. ($\sigma_c$)\vspace*{0.1cm}\end{minipage} & 
    \begin{minipage}{1.3cm}\centering Surround std. dev. ($\sigma_s$)\vspace*{0.1cm}\end{minipage} & 
    \begin{minipage}{1.3cm}\centering Sampling resolution (cols,rows)\vspace*{0.1cm}\end{minipage} \\
    \hline
    \begin{minipage}{0.7cm}\centering 1  \end{minipage} &
    \begin{minipage}{0.8cm}\centering off \vspace*{0.005cm} \end{minipage}& 
    \begin{minipage}{0.7cm}\centering$3$ \end{minipage}& 
    $0.8$ & $6.7 \times \sigma_c$ &  1, 1 \\
    \begin{minipage}{0.7cm}\centering 2 \end{minipage} & 
    \begin{minipage}{0.8cm}\centering on \vspace*{0.005cm}\end{minipage} & 
    \begin{minipage}{0.7cm}\centering $11$ \end{minipage}& 
    $1.04$ & $6.7 \times \sigma_c$ & 1, 1 \\
    \begin{minipage}{0.7cm}\centering3 \end{minipage} &
    \begin{minipage}{0.8cm}\centering off \vspace*{0.005cm}\end{minipage} & 
    \begin{minipage}{0.7cm}\centering $61$ \end{minipage}& 
    $8$ & $4.8 \times \sigma_c$ & 5, 3 \\
    \begin{minipage}{0.7cm}\centering 4  \end{minipage} & 
    \begin{minipage}{0.8cm}\centering on \vspace*{0.005cm}\end{minipage} & 
    \begin{minipage}{0.7cm}\centering $243$\end{minipage} &
    $10.4$ & $4.8 \times \sigma_c$ & 5, 3 
  \end{tabular}
  \egroup
 \end{center}
  \label{tab-kernel-specs}
\end{table}

Every pixel value in the convolved images (Fig. \ref{fig-convolution-results}) 
is inversely proportional to a spike emission time (i.e. the higher the pixel value, the sooner the spike will be sent out.)

\begin{figure}[hbt]
  \centering
  \subfloat[Original image]{
    \label{sfig-rank-ordered-original}
    \includegraphics[width=0.15\textwidth]{original_5_1}
  }
  \subfloat[Layer 1 (off-centre)]{
    \label{sfig-rank-ordered-midget-off}
    \includegraphics[width=0.15\textwidth]{filtered_focal_5_1_0}
  }
  \subfloat[Layer 2 (on-centre)]{
    \label{sfig-rank-ordered-midget-on}
    \includegraphics[width=0.15\textwidth]{filtered_focal_5_1_1}
  }\\
  \subfloat[Layer 3 (off-centre)]{
    \label{pic-lena-P-OFF}
    \includegraphics[width=0.15\textwidth]{filtered_focal_5_1_2}
  }
  \subfloat[Layer 4 (on-centre)]{
    \label{pic-lena-P-ON}
    \includegraphics[width=0.15\textwidth]{filtered_focal_5_1_3}
  }
  \subfloat[Reconstructed]{
    \label{pic-lena-reconstructed}
    \includegraphics[width=0.15\textwidth]{reconstructed_5_1}
  }
  \caption{Results of simulating ganglion cells (convolved images are enhanced for better contrast)}
  \label{fig-convolution-results}
\end{figure}
The algorithm also performs a redundancy correction step; it does so by 
adjusting the convolved image's pixel value according to the correlation 
between convolution kernels (Alg.~\ref{code-focal-corr}).
\begin{algorithm}[h]
  \caption{FoCal, Part 2}
  \label{code-focal-corr}
  \begin{algorithmic}
    \Procedure{Correction}{coeffs $C$, correlations $Q$}
    \State $N \leftarrow \emptyset$ \Comment{Corrected coefficients}
    \Repeat
    \State $m \leftarrow max(C)$\Comment{Obtain maximum from $C$}
    \State $M \leftarrow M \cup m$\Comment{Add maximum to $M$}
    \State $C \leftarrow C \setminus m$\Comment{Remove maximum from $C$}
    \ForAll{$ c \in C$} \Comment{Adjust all remaining $c$}
    \If{$Q(m, c) \neq 0$} \Comment{Adjust only near}
    \State $c \leftarrow c - m \times Q(m, c)$
    \EndIf
    \EndFor
    \Until{$C = \emptyset$}
    \State \textbf{return} $M$
    \EndProcedure
  \end{algorithmic}
\end{algorithm}
Two resolutions are provided for the rank-order encoded portion of the database, the first is the original $28\times28$ one. An additional up-scaled resolution version is also provided, the images where scaled to a $128\times128$ resolution using bi-cubic interpolation, this was done to match the DVS native resolution. The source Python scripts to transform images to ROC spike trains, to convert the results into AER and pyNN's spike source array will be provided.
	\subsubsection{DVS Sensor Output with Flashing Input}
	\subsubsection{DVS Sensor Output with Oscillating Input}
	\subsubsection{DVS Sensor Output with Moving Input}
	
\section{Performance Evaluation}
\label{sec:eval}

A crucial part of research is reporting results and comparing achievements with other state-of-the-art work. Unfortunately there is no standard way of fulfilling that task, this leads to a hard 
We would like to assist with the following considerations.


\subsection{Hardware-Independent}
A brief description of the \emph{network topology} is most welcome, we believe different topologies will have a deep impact on the overall performance. Furthermore, sharing this designs can inspire fellow scientists to create new structures that, might  a new point-of-view to bloom, generating a positive feedback loop where everybody wins.

Most classification papers report a percentage of accuracy that gives the reader a measure of the correct classifications~[\cite{dietterich1998approximate}]. Some times it might be desirable, for a better understanding of the paper, that a distinction between ambiguous, outliers and incorrect classes is made~[\cite{liu2002performance}]. A very useful piece of information is clear citation of the base-line source, which is almost always there but lost in a sea of references.

%Should we report also incorrect or ambiguous? Could some ``correct'' be masking ambiguous? Were the ambiguous due to noise? Was the noise added on purpose? 

As we are proposing spike based data-sets, it's desirable that the users specify if there was any preprocessing applied before actually feeding the spike trains into their networks~[\cite{best-practice-nn-img}]. For example, if we want to use a particular set to test noisy inputs, it would be extremely useful to have a notion of the type of noise added.

%Traditionally, neural network training has been done using rate-based encoding. As new theories emerge, a

An important distinction to make is the nature of the training procedure. One example is the way data was exposed to the network (e.g. How many times each image was presented? How much time was a single example shown?. [\cite{unsup_leraning_diehl}]) Also, details on the particulars of the implementation of the learning rule used (e.g. Delta, STDP, BCM, etc.) is highly desirable. 

One the biggest distinctions on learning procedures is whether they were done using some \emph{supervision} or not; making this distinction clear is vastly appreciated. On supervised learning, the label of the data influences to establish categories and connection weights. Unsupervised learning has fewer constraints when it comes to class creation but might be tougher to get right. 

A number of different classes are expected, this quantity might give an insight onto the network topology and dynamics. A description of the methods used to generate and populate the classes is very helpful for the reader. (e.g. Did we use a statistical measure? Was it a combination of NN with some other algorithms?)

\begin{table*}
  \caption{Hardware independent comparison}
  \begin{center}
    \bgroup
    \def\arraystretch{1.4}
    \begin{tabular}{ l | c c c c c c }
      $ $ &
      \begin{minipage}{1.9cm}Topology \end{minipage} & 
      \begin{minipage}{1.9cm}Accuracy \end{minipage} & 
      \begin{minipage}{1.9cm}Preprocessing \end{minipage} &
      \begin{minipage}{1.9cm}Training \end{minipage} & 
      \begin{minipage}{1.9cm}Supervised \end{minipage} &
      \begin{minipage}{1.9cm}Extra classifier \end{minipage} \\
      \hline
%contents
      \begin{minipage}{2cm} Paper 1 [ref] \end{minipage}  & & & & & & \\
      \begin{minipage}{2cm} Paper 2 [ref]\end{minipage}  & & & & & & \\
      \begin{minipage}{2cm} Paper 3 [ref]\end{minipage}  & & & & & & \\
      \begin{minipage}{2cm} Paper 4 [ref]\end{minipage}  & & & & & & 
    \end{tabular}
    \egroup
  \end{center}
  \label{tb:software_comparison}
\end{table*}


\subsection{Hardware-Specific}
Specifying hardware is of utmost importance when comparing computing times and power consumption. Analog or digital or a hybrid system.

Different platforms have special benefits, purely hardware solutions have low power consumption but lack programmability.

New theories, such as Polychronization, suggest that axonal delays are an integral part of the brain's computing mechanisms~[\cite{Izhikevich2005}]. 

Power consumption is a key issue for mobile applications and robotics. A way to measure is to state the number of \emph{Synaptic operations per Watt} that the hardware is capable of.

An important factor to measure performance is the number of \emph{Synaptic events per second}; i.e. rough throughput

A piece of hardware that is difficult to use/program is of little use, thus \emph{front-end support} is

Neural activity highly depends on synapses, specifying what model was used and its precision will impact on the performance.

\begin{table*}
  \caption{Hardware dependant comparison}
  \begin{center}
      \bgroup
      \def\arraystretch{1.4}
    \begin{tabular}{l | c c c c c c c c c}
      $ $ & 
       \begin{minipage}{1.2cm}\centering Hardware approach \end{minipage} & 
       \begin{minipage}{1.3cm}\centering Simulation type \end{minipage} & 
       \begin{minipage}{1.7cm}\centering Programmable \end{minipage} & 
       \begin{minipage}{1cm}\centering Axonal delays \end{minipage} & 
       \begin{minipage}{1cm}\centering Synaptic model \end{minipage} & 
       \begin{minipage}{1.2cm}\centering Synaptic precision \end{minipage} & 
       \begin{minipage}{1.2cm}\centering Synaptic events per sec \end{minipage} & 
       \begin{minipage}{1.4cm}\centering Synaptic ops per Watt \end{minipage} & 
       \begin{minipage}{1.7cm}\centering Programming front-end \end{minipage}  \\
       \hline
       % contents!
       \begin{minipage}{1.8cm}\centering Paper 1 [ref] \end{minipage} & & & & & & & & & \\
       \begin{minipage}{1.8cm}\centering Paper 2 [ref]\end{minipage} & & & & & & & & & \\
       \begin{minipage}{1.8cm}\centering Paper 3 [ref]\end{minipage} & & & & & & & & & \\
       \begin{minipage}{1.8cm}\centering Paper 4 [ref]\end{minipage} & & & & & & & & & 
    \end{tabular}
    \egroup
  \end{center}
  \label{tb:hardware_comparison}
\end{table*}
    
%table summary?

\section{Case Studies}
\label{sec:test}
In this chapter, we present two recognition systems on the Poissonian subset as an example use of the dataset.
\subsection{Case Study I}
Both the training and testing exploited the Poinssonian presentation of the digits.
The network was trained online with STDP learning rule.
The performance of the model was evaluated with both software simulation (on NEST~[\cite{gewaltig2007nest}]) and hardware implementation (on SpiNNaker). 
%\subsection{Description of the data used}
\subsubsection{Training}
There are two layers in the neural network model.
And 28$\times$28 input neurons fully connected to 100 output neurons.
Each output neuron represented a trained template of a digit.
Thus there were 10 templates for each digit.
The firing rate of the input neurons were assigned linearly according to their intensities and normalised with a total firing rate of 2000~$Hz$.
The training set of 60000 hand written digits were firstly classified into 100 classes, 10 classes per digit, using K-means clusters.
Every image was presented 300~$ms$ during training and at the same time a teaching signal of 50~$Hz$ was conveyed to the responding output neuron (1 out of 100) to trigger the learning, see Fig~\ref{Fig:train}.
\begin{figure}[hbt!]
	\centering
	\includegraphics[width=0.48\textwidth]{images/training.pdf}
	\caption{Training model of a single decision neuron.}
	\label{Fig:train}
\end{figure} 

The model utilised Leaky-Integrate-and-Fire (LIF) neurons, and the parameters were all with biological means, see the listed values in Table~\ref{tbl:pynnSetting}.
The trained weights were plotted in align with the input image size in Fig~\ref{Fig:weight}.
\begin{figure}[hbt!]
	\centering
	\includegraphics[width=0.48\textwidth]{images/weight_r.pdf}
	\caption{Trained weights of the synapses from input layer to output neurons.}
	\label{Fig:weight}
\end{figure}  

\begin{table}[hbbp]
\centering
\caption{\label{tbl:pynnSetting}Parameter setting for the current-based LIF neurons using PyNN}
\begin{tabular}{c|c|c}
\hline
Type & IF\_curr\_exp & Units\\
\hline
cm & 0.25 & nF	\\
%\hline
tau\_m & 20.0 & ms\\
%\hline
tau\_refrac & 2.0 & ms\\
%\hline
tau\_syn\_E & 1.0 & ms\\
%\hline
tau\_syn\_I & 1.0 & ms\\
%\hline
v\_reset & -70.0 & mV\\
%\hline
v\_rest & -65.0 & mV\\
%\hline
v\_thresh & -50.0 & mV\\
\hline
\end{tabular}
\end{table}

\subsubsection{Testing}
The weights were normalised after training and applied to the same network.
And weak weights (=0) were set to inhibitory connections with an identical strength.
%The output neurons inhibited all the other neurons as a winner take all circuit.
The feed=forward testing network is shown in Fig~\ref{Fig:test}.
Poissonion spike trains are generated with input neurons and conveyed through trained synaptic connections to the decision neurons.
Every testing image was presented 1 second to the network, and the output neuron with the highest firing rate decides.
The accuracy of the recognition reached 82.42\%.%83.14\%.
The recording of the output neurons of a test sequence of digits (4, 1, 1, 0, 9, 3, 1, 9, 4, 6) was shown in the raster plot (Fig~\ref{Fig:output}).
Fig~\ref{Fig:test}.
\begin{figure}[hbt!]
	\centering
	\includegraphics[width=0.48\textwidth]{images/testing.pdf}
	\caption{Testing model of the spiking neural network.}
	\label{Fig:test}
\end{figure} 
\begin{figure}[hbt!]
	\centering
	\includegraphics[width=0.48\textwidth]{images/test300-301.pdf}
	\caption{A raster plot of test of a digits sequence.}
	\label{Fig:output}
\end{figure} 
\subsubsection{Evaluation}

\subsection{[Evangelos Stromatias]Case Study II}
\section{Conclusion and the Future Work}
\label{sec:summ}
\subsection{What we have said and done}
%The Proposal: Advantages
In this work we propose an set of image conversion methods, in particular we have performed a full conversion of the MNIST database. Our implementations are open-source and can be obtained from a public repository.

In order to ease the understanding and comparison of investigation results, we suggest that researchers report typical neural networks characteristics as well as others that we believe are important (e.g. events per time unit, time per sample, response time).

The use of neuromorphic hardware in research is increasing, some of its characteristics may alter simulation results. We encourage researchers to describe some hardware characteristics that have a direct implication in the performance of neural networks.

\subsection{The future direction of developing the database}
What are the future algorithms we are encouraging.


\section*{Acknowledgments}
The work presented in this paper is largely inspired on discussions carried out at the 2015 Workshops on Neuromorphic Cognition Engineering in CapoCaccia.
The authors would like to thank the organisers and the sponsors.
The SpiNNaker project is supported by the Engineering and Physical Science Research Council (EPSRC grant EP/4015740/1), the EU Flagship Human Brain Project (FP7-604102), and also by ARM and Silistix.
The authors thank the support of these sponsors and industrial partners.
\bibliographystyle{frontiersinSCNS&ENG} % for Science and Engineering articles
%\bibliographystyle{frontiersinMED} % for Medicine articles

\bibliography{ref,rank-ordered,hw-ind-eval,hw-dep-eval}

\end{document}